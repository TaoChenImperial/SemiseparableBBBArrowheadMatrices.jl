\documentclass{article}
\usepackage{graphicx} % Required for inserting images
\usepackage{amsmath}
\usepackage{amsthm}
\newtheorem{theorem}{Theorem}
\newtheorem{lemma}{Lemma}
\title{A fast QR decomposition for Semiseparable plus banded matrices}
\author{Tao Chen}
\date{August 2025}

\begin{document}

\maketitle

Consider a semiseparable plus diagonal matrix $A$ with rank 1 in both the upper and lower semiseparable part,

\begin{equation}
A=
\begin{bmatrix}
d_1 & w_1s_2 & w_1s_3 & \cdots & w_1s_{n-1} & w_1s_n \\
u_2v_1 & d_2 & w_2s_3 & \cdots & w_2s_{n-1} & w_2s_n \\
u_3v_1 & u_3v_2 & d_3 & \cdots & w_3s_{n-1} & w_3s_n \\
\vdots & \vdots & \vdots & \ddots & \vdots & \vdots\\
u_{n-1}v_1 & u_{n-1}v_2 & u_{n-1}v_3 & \cdots & d_{n-1} & w_{n-1}s_n \\
u_nv_1 & u_nv_2 & u_nv_3 & \cdots & u_nv_{n-1} & d_n
\end{bmatrix}
\label{rank1matrix}
\end{equation}

It can be formulated as $A=tril(\mathbf{u}\mathbf{v}^T, -1) + D + triu(\mathbf{w}\mathbf{s}^T, 1)$ where $\mathbf{u}=(u_1,...,u_n)^T$, $\mathbf{v}=(v_1,...,v_n)^T$, $\mathbf{w}=(w_1,...,w_n)^T$, $\mathbf{s}=(s_1,...,s_n)^T$, and $D=diag(d_1, d_2, ..., d_n)$. 

After applying QR decomposition to $A$, the resulting factor matrix $F$ retains a semiseparable plus diagonal structure. Specifically, the lower semiseparable part has rank 1, while the upper semiseparable part has rank 2. The upper triangular portion of $F$ stores the matrix $R$, and the lower triangular portion contains the Householder reflection vectors $\mathbf{y}$ generated during the decomposition process.

Next, we demonstrate by induction why the factor matrix $F$ mantains such a semiseparable plus diagonal structure. 

Before we start, an important notation will be: for a matrix $S$, let $S[i:j,m:n]$ represent the submatrix of $S$ from row $i$ to row $j$ and from column $m$ to column $n$. When $i=j$ or $m=n$, the notation will be simplified as $S[i,m:n]$ or $S[i:j,m]$.

Let's first introduce a very helpful lemma:

\begin{lemma} \label{helpful_lemma}
Suppose an $n$ by $n$ matrix 

$$B:=tril(\mathbf{u}\mathbf{v}^T, -1) + D + triu(\mathbf{w}\mathbf{s}^T, 1)$$

where $\mathbf{u}=(u_1,...,u_n)^T$, $\mathbf{v}=(v_1,...,v_n)^T$, $\mathbf{w}=(w_1,...,w_n)^T$, $\mathbf{s}=(s_1,...,s_n)^T$, and $D=diag(d_1, d_2, ..., d_n)$. Let 

$$C:=B+\mathbf{u}(q\mathbf{s}'+k\mathbf{u}'B)$$ 

where $q$ and $k$ are two coefficients. Also let 

$$\tilde{C}:=(I-\tau\mathbf{y}\mathbf{y}')C$$

where $\tau$ is a coefficient and $\mathbf{y}=\mathbf{e}_1+\bar{k}\mathbf{u}^{(2)}$ with $\mathbf{e}_1=(1,0,...,0)^T\in R^n$, $\mathbf{u}^{(2)}=(0,u_2,u_3,...,u_n)^T$, and $\bar{k}$ a coefficient. At last define 

$$\gamma:=-\tau q\mathbf{e'}_1\mathbf{u}-\tau q\bar{k}\mathbf{u}^{(2)'}\mathbf{u},$$

$$\phi:=-\tau q\bar{k}\mathbf{e'}_1\mathbf{u}-\tau q\bar{k}^2\mathbf{u}^{(2)'}\mathbf{u},$$

$$\xi:=-\tau k\mathbf{e'}_1\mathbf{u}-\tau k\bar{k}\mathbf{u}^{(2)'}\mathbf{u},$$

and 

$$\eta:=-\tau k\bar{k}\mathbf{e'}_1\mathbf{u}-\tau k \bar{k}^2\mathbf{u}^{(2)'}\mathbf{u}.$$

Then we have:

\textbf{firstly}, $\tilde{C}[2:n,2:n]$ has the same form as $C$. Specifically, let $\tilde{B}=B[2:n,2:n], \tilde{\mathbf{u}}=[u_2,u_3,...,u_n]\in R^{n-1}$, and $\tilde{\mathbf{s}}=[s_2,s_3,...,s_n]\in R^{n-1}$, then

\begin{equation}
    \tilde{C}[2:n,2:n]=\tilde{B}+\tilde{\mathbf{u}}(\tilde{q}\tilde{\mathbf{s}}+\tilde{k}\tilde{\mathbf{u}}'\tilde{B})
\end{equation}
where
\begin{equation}
 \tilde{q}=q+ku_1w_1-\bar{k}\tau w_1+\phi+\eta u_1w_1
\end{equation}
and
\begin{equation}
    \tilde{k}=k-\bar{k}^2\tau+\eta.
\end{equation}

\textbf{Secondly}, let $\bar{\mathbf{s}}=\mathbf{u}'B$, then
\begin{equation}
    \tilde{C}[1,2:n]=\alpha \mathbf{s}[2:n]+\beta\bar{\mathbf{s}}[2:n]
\end{equation}
where
\begin{equation}
    \alpha=w_1+qu_1+ku_1^2w_1-\tau w_1+\gamma - ku_1^2w_1+\tau \bar{k}w_1u_1
\end{equation}
and
\begin{equation}
    \beta=ku_1-\tau \bar{k}+\xi.
\end{equation}

\end{lemma}

\begin{proof}
\begin{equation}
    \begin{aligned}
        \tilde{C}=&(I-\tau\mathbf{y}\mathbf{y}')(B+\mathbf{u}(q\mathbf{s}'+k\mathbf{u}'B))\\
        =&(I-\tau(\mathbf{e}_1+\bar{k}\mathbf{u}^{(2)})(\mathbf{e'}_1+\bar{k}\mathbf{u}^{(2)'}))(B+\mathbf{u}(q\mathbf{s}'+k\mathbf{u}'B))\\
        =&B+\mathbf{u}(q\mathbf{s}'+k\mathbf{u}'B)-\tau(\mathbf{e}_1+\bar{k}\mathbf{u}^{(2)})(\mathbf{e'}_1+\bar{k}\mathbf{u}^{(2)'})B\\
        &-\tau(\mathbf{e}_1+\bar{k}\mathbf{u}^{(2)})(\mathbf{e'}_1+\bar{k}\mathbf{u}^{(2)'})\mathbf{u}(q\mathbf{s}'+k\mathbf{u}'B)\\
        =&B+q\mathbf{u}\mathbf{s}'+k\mathbf{u}\mathbf{u}'B-\tau\mathbf{e}_1\mathbf{e'}_1B\\
        &-\tau\bar{k}\mathbf{e}_1\mathbf{u}^{(2)'}B-\bar{k}\tau \mathbf{u}^{(2)}\mathbf{e'}_1B-\bar{k}^2\tau\mathbf{u}^{(2)}\mathbf{u}^{(2)'}B\\
        &-\tau q\mathbf{e}_1\mathbf{e'}_1\mathbf{u}\mathbf{s'}-\tau q\bar{k}\mathbf{e}_1\mathbf{u}^{(2)'}\mathbf{u}\mathbf{s}'\\
        &-\tau q\bar{k}\mathbf{u}^{(2)}\mathbf{e'}_1\mathbf{u}\mathbf{s}'-\tau q\bar{k}^2\mathbf{u}^{(2)}\mathbf{u}^{(2)'}\mathbf{u}\mathbf{s}'\\
        &-\tau k\mathbf{e}_1\mathbf{e'}_1\mathbf{u}\mathbf{u}'B-\tau k\bar{k}\mathbf{e}_1\mathbf{u}^{(2)'}\mathbf{u}\mathbf{u}'B\\&-\tau k\bar{k}\mathbf{u}^{(2)}\mathbf{e'}_1\mathbf{u}\mathbf{u}'B-\tau k\bar{k}^2\mathbf{u}^{(2)}\mathbf{u}^{(2)'}\mathbf{u}\mathbf{u}'B\\
        =&B+q\mathbf{u}\mathbf{s}'+k\mathbf{u}\mathbf{u}'B-\tau\mathbf{e}_1\mathbf{e'}_1B\\
        &-\tau\bar{k}\mathbf{e}_1\mathbf{u}^{(2)'}B-\bar{k}\tau \mathbf{u}^{(2)}\mathbf{e'}_1B-\bar{k}^2\tau\mathbf{u}^{(2)}\mathbf{u}^{(2)'}B\\
        &+\gamma \mathbf{e}_1\mathbf{s}+\phi \mathbf{u}^{(2)}\mathbf{s}'+\xi\mathbf{e}_1\mathbf{u}'B+\eta\mathbf{u}^{(2)}\mathbf{u}'B\\
        =&B+q(u_1\mathbf{e}_1+\mathbf{u}^{(2)})\mathbf{s}'+k(u_1\mathbf{e}_1+\mathbf{u}^{(2)'})(u_1\mathbf{e'}_1+\mathbf{u}^{(2)'})\mathbf{u}'B\\
        &-\tau\mathbf{e}_1\mathbf{e'}_1B-\tau\bar{k}\mathbf{e}_1\mathbf{u}^{(2)'}B-\bar{k}\tau \mathbf{u}^{(2)}\mathbf{e'}_1B-\bar{k}^2\tau\mathbf{u}^{(2)}\mathbf{u}^{(2)'}B\\
        &+\gamma \mathbf{e}_1\mathbf{s}+\phi \mathbf{u}^{(2)}\mathbf{s}'+\xi\mathbf{e}_1(u_1\mathbf{e'}_1+\mathbf{u}^{(2)'})B+\eta\mathbf{u}^{(2)}(u_1\mathbf{e'}_1+\mathbf{u}^{(2)'})B\\
        =&B+qu_1\mathbf{e}_1\mathbf{s}'+q\mathbf{u}^{(2)}\mathbf{s}'+ku_1^2\mathbf{e}_1\mathbf{e'}_1B+ku_1\mathbf{e}_1\mathbf{u}^{(2)'}B+ku_1\mathbf{u}^{(2)}\mathbf{e'}_1B\\
        &+k\mathbf{u}^{(2)}\mathbf{u}^{(2)'}B-\tau\mathbf{e}_1\mathbf{e'}_1B-\tau\bar{k}\mathbf{e}_1\mathbf{u}^{(2)'}B-\bar{k}\tau \mathbf{u}^{(2)}\mathbf{e'}_1B\\
        &-\bar{k}^2\tau\mathbf{u}^{(2)}\mathbf{u}^{(2)'}B+\gamma \mathbf{e}_1\mathbf{s}+\phi \mathbf{u}^{(2)}\mathbf{s}'+\xi u_1\mathbf{e}_1\mathbf{e'}_1B+\xi\mathbf{e}_1\mathbf{u}^{(2)'}B\\
        &+\eta u_1\mathbf{u}^{(2)}\mathbf{e'}_1B+\eta\mathbf{u}^{(2)}\mathbf{u}^{(2)'}B
    \end{aligned}
\end{equation}

In the above results, we classify that there are 5 types of terms:

1. $B$, 

2. $\mathbf{u}^{(2)}\mathbf{s}'$(or $\mathbf{u}^{(2)}\mathbf{e'}_1B$ b.c. $(\mathbf{u}^{(2)}\mathbf{e'}_1B)[2:n, 2:n]=(w_1\mathbf{s}')[2:n, 2:n]$), 

3. $\mathbf{u}^{(2)}\mathbf{u}^{(2)'}B$, 

4. $\mathbf{e}_1\mathbf{s}'$(or $\mathbf{e}_1\mathbf{e'}_1B$ b.c. $(\mathbf{e}_1\mathbf{e'}_1B)[1,2:n]=(w_1\mathbf{e}_1\mathbf{s}')[1,2:n]$), and 

5. $\mathbf{e}_1\mathbf{u}^{(2)}B$

Combining terms of type 1, 2, and 3 we get that

\begin{equation}
\tilde{C}[2:n,2:n]=\tilde{B}+\tilde{\mathbf{u}}(\tilde{q}\tilde{\mathbf{s}}+\tilde{k}\tilde{\mathbf{u}}'\tilde{B})
\end{equation}

where

\begin{equation}
 \tilde{q}=q+ku_1w_1-\bar{k}\tau w_1+\phi+\eta u_1w_1
\end{equation}
and
\begin{equation}
    \tilde{k}=k-\bar{k}^2\tau+\eta;
\end{equation}

combining terms of type 1, 4, and 5 we get that 

\begin{equation}
\begin{aligned}
    \tilde{C}[1,2:n]=&(w_1+qu_1+ku_1^2w_1-\tau w_1+\gamma +\xi u_1w_1) \mathbf{s}[2:n]\\&+(ku_1-\tau \bar{k}+\xi)(\mathbf{u}^{(2)'}B)[2:n].\label{first_row_Ctilde}
\end{aligned}
\end{equation}

Since $(\mathbf{u}^{(2)'}B)[2:n]=(\mathbf{u}'B-w_1u_1\mathbf{s}')[2:n]=(\bar{\mathbf{s}}'-w_1u_1\mathbf{s}')[2:n]$, we can also write Eq.(\ref{first_row_Ctilde}) as
\begin{equation}
    \tilde{C}[1,2:n]=\alpha \mathbf{s}[2:n]+\beta\bar{\mathbf{s}}[2:n]
\end{equation}
where
\begin{equation}
    \alpha=w_1+qu_1+ku_1^2w_1-\tau w_1+\gamma - ku_1^2w_1+\tau \bar{k}w_1u_1
\end{equation}
and
\begin{equation}
    \beta=ku_1-\tau \bar{k}+\xi.
\end{equation}

\end{proof}


With this lemma, we can now prove the following theorem.


\begin{theorem}
After applying QR decomposition to (\ref{rank1matrix}), the factor matrix $F$ obtained is a semiseparable plus diagonal matrix. Its lower semiseparable part has rank 1 and upper semiseparable part has rank 2. 
\end{theorem}

\begin{proof}

The QR decomposition can be done by performing a series of Householder Transformations(HT) to eliminate elements below the diagonal of $A$ in the first column, the second column, ..., up to the nth column in order.

First, we perform an HT to eliminate the first column, that is, computing $(I-\tau_1 \mathbf{y}_1 \mathbf{y}^T_1)A$, where $\mathbf{y}_1$ is a regularized reflection vector s.t. the first nonzero element in $\mathbf{y}_1$ is $1$, and $\tau_1$ is a coefficient found during the process of the first HT. It is easy to see that $\mathbf{y}_1$ can be represented as $\mathbf{y}_1=\mathbf{e}_1+\bar{k}_2\mathbf{u}^{(2)}$ where $\mathbf{e}_1=[1,0,...,0]^T \in R^n$, $\mathbf{u}^{(2)}=[0,u_2,...,u_n]^T \in R^n$. The first column of $F$ below the diagonal can be determined as
\begin{equation}
F[2:n, 1]=\bar{k}_2\mathbf{u}[2:n].
\end{equation}

Define $A_i=A[i:n,i:n]\in R^{n+1-i,n+1-i}$, $\mathbf{u}_i=[u_i,u_{i+1,...,u_n}]^T\in R^{n+1-i}$, and $\mathbf{s}_i=[s_i,s_{i+1},...,s_n]^T\in R^{(n+1-i)}$. Let $A^{(i)}$ be matrix $A$ after the $i$th HT, we have $A^{(1)}:=(I-\tau_1 \mathbf{y}_1 \mathbf{y}^T_1)A$ now. Also let $\bar{\mathbf{s}}:=\mathbf{u}'A$, then apply lemma (\ref{helpful_lemma}) by setting $B=A$ and $q=k=0$, we have:

i.

\begin{equation}
A ^{(1)}[2:n,2:n]=A_2+\mathbf{u}_2(q_2\mathbf{s}_2+k_2\mathbf{u}'_2A_2)
\end{equation}
with 
\begin{equation}
q_2=-\bar{k}_2\tau_1w_1
\end{equation}

and
\begin{equation}
    k_2=-\bar{k}_2^2\tau_1
\end{equation}

ii.

\begin{equation}
    F[1,2:n]=A^{(1)}[1,2:n]=\alpha_2 \mathbf{s}[2:n]+\beta_2 \bar{s}[2:n]
\end{equation}
with
\begin{equation}
    \alpha_2=w_1-\tau_1 w_1+\tau_1 \bar{k}_2w_1u_1
\end{equation}
and
\begin{equation}
    \beta_2=-\tau_1\bar{k}_2
\end{equation}


\textbf{Next is the induction part}:

Suppose that after the $j$th HT($1\leq j<n$), $F[i+1:n, i]$ and $F[i,i+1:n]$ are both determined as

\begin{equation}
    F[i+1:n, i]=\bar{k}_{i+1}\mathbf{u}[i+1:n],\label{F_tril}
\end{equation}

\begin{equation}
    F[i,i+1:n]=\alpha_{i+1}\mathbf{s}[i+1:n]+\beta_{i+1}\bar{\mathbf{s}}[i+1:n] \label{F_triu}
\end{equation}

for $i=1,..,j$. Here $\bar{k}_i$, $\alpha_i$, and $\beta_i$ are coefficients determined during the previous HTs.

Also suppose that 

\begin{equation}
    A^{(j)}[j+1:n,j+1:n]=A_{j+1}+\mathbf{u}_{j+1}(q_{j+1}\mathbf{s}_{j+1}+k_{j+1}\mathbf{u}'_{j+1}A_{j+1})\label{After_HT}
\end{equation}

If $j+1<n$, we further perform the next $j+1$th HT, i.e., we multiply $I-\tau_{j+1} \mathbf{y}_{j+1} \mathbf{y}_{j+1}^T$ on the submatrix $A^{(j)}[j+1:n,j+1:n]$, where $\mathbf{y}_{j+1}$ is the regularized reflection vector for the $j+1$th HT and $\tau_{j+1}$ is a coefficient found during the $j+1$th HT. It is easy to see that $\mathbf{y}_{j+1}$ can be represented as $\mathbf{y}_{j+1}=\mathbf{e}_{j+1}+\bar{k}_{j+2}\mathbf{u}^{(j+2)}$ where 

$$
\mathbf{e}_{j+1}=[1,0,...,0]^T\in R^{n-j},
$$ 
$$
\mathbf{u}^{(j+2)}=[0, u_{j+2},...,u_n]^T\in R^{n-j}
$$.

and $\bar{k}_{j+2}$ is a coefficient found during the $j+1$th HT. 

Therefore, when $j<n-1$, the $j+1$th column of $F$ below the diagonal can be updated as:

\begin{equation}
    F[j+2:n,j+1]=\bar{k}_{j+2}\mathbf{u}[j+2:n].\label{F_tril_2}
\end{equation}

Now apply lemma(\ref{helpful_lemma}) by setting $C=A^{(j)}[j+1:n,j+1:n]$ and precompute 4 coefficients  

$$\gamma_{j+1}:=-\tau_{j+1} q_{j+1}\mathbf{e'}_{j+1}\mathbf{u}_{j+1}-\tau_{j+1} q_{j+1}\bar{k}_{j+2}\mathbf{u}^{(j+2)'}\mathbf{u}_{j+1},$$

$$\phi_{j+1}:=-\tau_{j+1} q_{j+1}\bar{k}_{j+2}\mathbf{e'}_{j+1}\mathbf{u}_{j+1}-\tau_{j+1} q_{j+1}\bar{k}_{j+2}^2\mathbf{u}^{(j+2)'}\mathbf{u}_{j+1},$$

$$\xi_{j+1}:=-\tau_{j+1} k_{j+1}\mathbf{e'}_{j+1}\mathbf{u}_{j+1}-\tau_{j+1} k_{j+1}\bar{k}_{j+2}\mathbf{u}^{(j+2)'}\mathbf{u}_{j+1},$$

and 

$$\eta_{j+1}:=-\tau_{j+1} k_{j+1}\bar{k}_{j+2}\mathbf{e'}_{j+1}\mathbf{u}_{j+1}-\tau_{j+1} k_{j+1} \bar{k}_{j+2}^2\mathbf{u}^{(j+2)'}\mathbf{u}_{j+1}.$$

We get:

i.

\begin{equation}
    A^{(j+1)}[j+2:n,j+2:n]=A_{j+2}+\mathbf{u}_{j+2}(q_{j+2}\mathbf{s}_{j+2}+k_{j+2}\mathbf{u'}_{j+2}A_{j+2})\label{After_HT_2}
\end{equation}
with
\begin{equation}
    q_{j+2}=q_{j+1}+k_{j+1}u_{j+1}w_{j+1}-\bar{k}_{j+2}\tau_{j+1} w_1+\phi_{j+1}+\eta_{j+1}u_{j+1}w_{j+1}
\end{equation}
and
\begin{equation}
    k_{j+2}=k_{j+1}-\bar{k}_{j+2}^2\tau_{j+1}+\eta_{j+1}
\end{equation}

ii.

\begin{equation}
\begin{aligned}
    F[j+1,j+2:n]&=A^{(j+1)}[j+1,j+2:n]\\&=\tilde{\alpha}_{j+2} \mathbf{s}[j+2:n]+\tilde{\beta}_{j+2}(\mathbf{u}_{j+1}A_{j+1})[j+2:n] \label{rewrite_needed}
\end{aligned}
\end{equation}
with
\begin{equation}
\begin{aligned}
    \tilde{\alpha}_{j+2}=&w_{j+1}+q_{j+1}u_{j+1}+k_{j+1}u_{j+1}^2w_{j+1}-\tau_{j+1} w_{j+1}\\
    &+\gamma_{j+1} - k_{j+1}u_{j+1}^2w_{j+1}+\tau_{j+1} \bar{k}_{j+2}w_{j+1}u_{j+1}
\end{aligned}
\end{equation}
and
\begin{equation}
    \tilde{\beta}_{j+2}=k_{j+1}u_{j+1}-\tau_{j+1} \bar{k}_{j+2}+\xi_{j+1}.
\end{equation}

Since 
\begin{equation}
    (\mathbf{u}_{j+1}A_{j+1})[j+2:n]=\bar{\mathbf{s}}[j+2:n]-(\sum_{l=1}^ju_lw_l)\mathbf{s}[j+2:n],
\end{equation}

Eq.(\ref{rewrite_needed}) can also be rewritten as:

\begin{equation}
    F[j+1,j+2:n]=\alpha_{j+2} \mathbf{s}[j+2:n]+\beta_{j+2}\bar{\mathbf{s}}[j+2:n] \label{F_triu_2}
\end{equation}
with
\begin{equation}
\begin{aligned}
    \alpha_{j+2}=&w_{j+1}+q_{j+1}u_{j+1}+k_{j+1}u_{j+1}^2w_{j+1}-\tau_{j+1} w_{j+1}+\gamma_{j+1}\\& - k_{j+1}u_{j+1}^2w_{j+1}+\tau_{j+1} \bar{k}_{j+2}w_{j+1}u_{j+1}\\&+(-k_{j+1}u_{j+1}+\tau_{j+1} \bar{k}_{j+2}-\xi_{j+1})\epsilon_j
\end{aligned}
\end{equation}
where
\begin{equation}
    \epsilon_j=\sum_{l=1}^ju_lw_l,
\end{equation}
and
\begin{equation}
    \beta_{j+2}=k_{j+1}u_{j+1}-\tau_{j+1} \bar{k}_{j+2}+\xi_{j+1}.
\end{equation}

After the $j+1$th HT, (\ref{F_tril_2}) has the same form as (\ref{F_tril}), (\ref{F_triu_2}) the same form as (\ref{F_triu}), and (\ref{After_HT_2}) the same form as (\ref{After_HT}). Therefore, by induction, we have $F[i+1:n, i]=\bar{k}_{i+1}\mathbf{u}[i+1:n]$ and $F[i,i+1:n]=\alpha_{i+1}\mathbf{s}[i+1:n]+\beta_{i+1}\bar{\mathbf{s}}[i+1:n]$ for $i=1,...,n-1$. Formally,

\begin{equation}
    F=tril(\mathbf{u}\bar{\mathbf{k}}^T,-1)+D_F+triu(\mathbf{a}\mathbf{s}^T+\mathbf{b}\bar{s}^T,1)
\end{equation}

where $\bar{\mathbf{k}}=[\bar{k}_2,...,\bar{k}_n,0]^T\in R^n$, $\mathbf{a}=[\alpha_2,...\alpha_n,0]^T\in R^n$, $\mathbf{b}=[\beta_2,...,\beta_n, 0]\in R^n$, and $D_F$ a diagonal matrix s.t.

\begin{equation}
D_F[i,i]=
\begin{cases}
    A^{(i)}[i,i], & i=1,2,...,n-1 \\
    A^{(n-1)}[n,n], & i=n
\end{cases}
\end{equation}

\end{proof}



\end{document}
